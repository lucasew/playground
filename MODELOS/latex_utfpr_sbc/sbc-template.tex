\documentclass[12pt]{article}

\usepackage{sbc-template}

\usepackage{graphicx,url}
\usepackage{lipsum}

\usepackage[brazil]{babel}   
%\usepackage[latin1]{inputenc}  
\usepackage[utf8]{inputenc}  
% UTF-8 encoding is recommended by ShareLaTex

     
\sloppy

\title{Aplicação de jogo do bicho para ensino de economia}

\author{Lucas Eduardo Wendt\inst{1} }


\address{Bacharelado em Ciência da Computação -- Universidade Tecniológica Federal do Paraná
  (UTFPR)\\
  85892-000 -- Santa Helena -- PR -- Brazil
  \email{lucaswendt@alunos.utfpr.edu.br}
}

\begin{document} 

\maketitle

\begin{abstract}
  % todo: fazer
  This meta-paper describes the style to be used in articles and short papers
  for SBC conferences. For papers in English, you should add just an abstract
  while for the papers in Portuguese, we also ask for an abstract in
  Portuguese (``resumo''). In both cases, abstracts should not have more than
  10 lines and must be in the first page of the paper.
\end{abstract}
     
\begin{resumo} 
  Este meta-artigo descreve o estilo a ser usado na confecção de artigos e
  resumos de artigos para publicação nos anais das conferências organizadas
  pela SBC. É solicitada a escrita de resumo e abstract apenas para os artigos
  escritos em português. Artigos em inglês deverão apresentar apenas abstract.
  Nos dois casos, o autor deve tomar cuidado para que o resumo (e o abstract)
  não ultrapassem 10 linhas cada, sendo que ambos devem estar na primeira
  página do artigo.
\end{resumo}

% introdução por ultimo
% objetivo -> fica na introdução
% metodologia
\section{Introdução}

\lipsum[2]

\section{Materiais e Métodos}

\lipsum[3]

\section{Resultados e conclusões}

\lipsum[4]

\begin{figure}[ht]
\centering
\includegraphics[width=.5\textwidth]{fig1.jpg}
\caption{A typical figure}
\label{fig:exampleFig1}
\end{figure}

\begin{figure}[ht]
\centering
\includegraphics[width=.3\textwidth]{fig2.jpg}
\caption{This figure is an example of a figure caption taking more than one
  line and justified considering margins mentioned in Section~\ref{sec:figs}.}
\label{fig:exampleFig2}
\end{figure}


\begin{table}[ht]
\centering
\caption{Variables to be considered on the evaluation of interaction
  techniques}
\label{tab:exTable1}
\smallskip
\begin{tabular}{|l|c|c|}
\hline
& Value 1 & Value 2\\[0.5ex]
\hline
&&\\[-2ex]
Case 1 & 1.0 $\pm$ 0.1 & 1.75$\times$10$^{-5}$ $\pm$ 5$\times$10$^{-7}$\\[0.5ex]
\hline
&&\\[-2ex]
Case 2 & 0.003(1) & 100.0\\[0.5ex]
\hline
\end{tabular}
\end{table}

\section{Referências Bibliográficas}
Bibliographic references must be unambiguous and uniform.  We recommend giving
the author names references in brackets, e.g. \cite{knuth:84},
\cite{boulic:91}, and \cite{smith:99}.

The references must be listed using 12 point font size, with 6 points of space
before each reference. The first line of each reference should not be
indented, while the subsequent should be indented by 0.5 cm.

\bibliographystyle{sbc}
\bibliography{sbc-template}

\end{document}

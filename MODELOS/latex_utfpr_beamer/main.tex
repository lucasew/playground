% ------------------------------------------------------------------------
% TCC: Modelo de Apresentação de Trabalho Monográfico Acadêmico  
% 
% Template v2.5 por Francisco Reinaldo 
%		(https://orcid.org/0000-0001-6161-6755)
%		(http://lattes.cnpq.br/7401534350061823)
% Ajustes extra by Lucas Eduardo Wendt
% Adaptado da versão 1, em aula, de: Rodrigo de Moraes
% ------------------------------------------------------------------------
% Agradecimentos a Overleaf pela oportunidade 
% ------------------------------------------------------------------------
\documentclass[13pt,xcolor=table]{beamer}
\usepackage{background,quoting,setspace}
\usepackage[utf8]{inputenc}
\usepackage[T1]{fontenc}
\usepackage[brazil]{babel}
\usepackage{graphicx}
\usepackage[many]{tcolorbox}
%\usepackage[svgnames,dvipsnames,x11names,table,xcdraw]{xcolor} %para tabelas com cor de fundo 
\usepackage{lmodern}% or any other vector / postscript font
\usepackage{lipsum}

\setbeamercolor{structure}{fg=black}
\setbeamercolor{normal text}{fg=black,bg=white}
\setbeamertemplate{background canvas}{\includegraphics[width=\paperwidth,height=\paperheight]{Imagem.png}}

\newcommand{\cabecalho}[2]{\vspace*{\fill}\LARGE\bfseries{\begin{center} {#1}\\ \normalsize {#2} \end{center}}\par}

\newcommand{\titulotrabalho}[1]{\vspace*{\fill}\large \bfseries{\centering\MakeUppercase{#1}\par}}

\newcommand{\autor}[1]{\vspace*{\fill}\large \bfseries\hspace{\fill}{#1}\par}

\newcommand{\orientadores}[2]{\vspace*{26pt}\normalsize \bfseries\hspace{\fill}{#1}\par\hspace{\fill}{#2}\vspace*{\fill}}



    \newenvironment{lamina}[1]
    {
    \vspace*{5em}
    \begin{minipage}[t][7cm][t]{\textwidth}
    {\LARGE\bfseries\centering\MakeUppercase{#1}\par\vspace{1em}\normalsize}
    }
    {
    \vspace{\fill}\end{minipage}
    }
    
\def\pinta{\cellcolor[HTML]{9B9B9B}}

\newcommand{\caixa}[1]{\fbox{\parbox{\dimexpr\linewidth-2\fboxsep-2\fboxrule\relax}{\centering {#1} }}}



\begin{document}

%---------------------------------------

\begin{frame} %Slide CAPA
	\cabecalho{Curso}{Departamento}
	\titulotrabalho{Título}
	\autor{Autor}
	\orientadores{Orientador: Prof. Beltrano}{}
\end{frame}


%---------------------------------------

\begin{frame}
	\begin{lamina}{Definição do Problema}
		Alunos com
		\begin{itemize}
			\item dificuldade em operações básicas
			\item alta taxa de reprovação em Matemática
			\item pouca utilização de ferramentas tecnológicas
			\item baixa interatividade coletiva
		\end{itemize}
	\end{lamina}
\end{frame}

%-------------------------------------

\begin{frame}
	\begin{lamina}{Objetivo Geral}
		\vspace{3em}
    		\caixa{Atrair o interesse dos alunos \\em expressões matemáticas}
	\end{lamina}
\end{frame}

%-------------------------------------

\begin{frame}
	\begin{lamina}{Objetivo Específicos}
		\begin{itemize}
			\item Levantar dificuldades dos alunos em compreender os conceitos matemáticos
			\item Utilizar Gamificação
			\item Desenvolver um jogo que ajude a ensinar as quatro operações básicas da matemática
			\item Demonstrar a aplicação do conceito de gamificação em um {\it software} educacional
		\end{itemize}
	\end{lamina}
\end{frame}

%-------------------------------------

\begin{frame}
	\begin{lamina}{Justificativa}
		\begin{itemize}
			\item Gamificação desperta atenção do aluno
			\item O professor participará como mediador das atividades gamificadas
			\item A gamificação utiliza-se de matemática a todo momento
		\end{itemize}
	\end{lamina}
\end{frame}

%-------------------------------------

\begin{frame}
	\begin{lamina}{Materiais e Métodos}
		\begin{itemize}
			\item Ferramenta para implementação: Godot
			\item Método de desenvolvimento: Cascata
		\end{itemize}
	\end{lamina}
\end{frame}

%-------------------------------------

\begin{frame}
	\begin{lamina}{Cronograma}
		
		
		\begin{tabular}{|p{4.5cm}|p{1.4em}|p{1.4em}|p{1.4em}|p{1.4em}|p{1.4em}|p{1.4em}|}
			\hline
			\textbf{Etapas}                          & \textbf{Fev} & \textbf{Mar} & \textbf{Abr} & \textbf{Mai} & \textbf{Jun} & \textbf{Jul} \\ \hline
			Levantamento de Dados                    & \pinta       & \pinta       &              &              &              &              \\ \hline
			Identificar Literatura Disponível       &              & \pinta       & \pinta       & \pinta       &              &              \\ \hline
			Organizar Informações                  &              & \pinta       & \pinta       & \pinta       &              &              \\ \hline
			Prototipar o \textit{Software }          &              &              & \pinta       & \pinta       &              &              \\ \hline
			Elaborar Mini Capítulos para o Software &              &              & \pinta       & \pinta       &              &              \\ \hline
			Concluir o\textit{ Software }            &              &              &              & \pinta       & \pinta       & \pinta       \\ \hline
		\end{tabular}
		
	\end{lamina}
\end{frame}

%-----------------------------------------

\begin{frame}
	\begin{lamina}{Resultados Esperados}
		\begin{itemize}
			\item Permitir aumentar o interesse pela disciplina em questão
			\item Permitir ampliar a comunicação entre professor e aluno
			\item Familiarizar ao professor o uso de ferramentas tecnológicas em sala de aula
			\item Multiplataforma
		\end{itemize}
	\end{lamina}
\end{frame}

%-------------------------------------

\begin{frame}<presentation:0>
	\nocite{d1996educaccao}
\end{frame}


\begin{frame}
	\begin{lamina}{Bibliografia}
		\bibliographystyle{plainnat}
		\bibliography{bib}
	\end{lamina}
\end{frame}

%--------------------------------------------

\begin{frame}
	\begin{lamina}{Agradecimentos}
		\begin{itemize}\Large
			\item Ao Orientador, Prof. Doutor Eng. Francisco Antonio Fernandes Reinaldo
			\item Aos membros da banca, Profs. MSc. Maicon Felipe Malacarne e Prof. MSc. Celso Hotz
		\end{itemize}
		\vspace{2pt}
		\centering \includegraphics[keepaspectratio=true,scale=0.25]{overleaf}
	\end{lamina}
\end{frame}


\end{document}

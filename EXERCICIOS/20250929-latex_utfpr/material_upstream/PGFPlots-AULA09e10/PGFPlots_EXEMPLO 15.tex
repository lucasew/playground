\textbf{\uline{Exemplo 15: Gráfico de setores padrão}}	
	\begin{center}
		\begin{tikzpicture}[scale=0.5]
			\pie{35/Palmeiras,
				26/São Paulo,
				30/Corinthians,
				9/Santos}
		\end{tikzpicture}
	\end{center}
	
\textbf{\uline{Exemplo 15: Gráfico de setores com raio alterado}}
    \begin{center}
		\begin{tikzpicture}[scale=0.5]
			\pie[radius = 3.3, rotate = 90]{35/Palmeiras,
				26/São Paulo,
				30/Corinthians,
				9/Santos}
		\end{tikzpicture}
	\end{center}
	
\textbf{\uline{Exemplo 15: Gráfico de setores com mudança de cores}}
    \begin{center}
		\begin{tikzpicture}[scale=0.5]
			\pie[color = {yellow, red, blue, green}]{35/Palmeiras,
				26/São Paulo,
				30/Corinthians,
				9/Santos}
		\end{tikzpicture}
	\end{center}
	
\textbf{\uline{Exemplo 15: Gráfico de setores com setores deslocados}}
    \begin{center}
		\begin{tikzpicture}[scale=0.5]
			\pie[explode = 0.1]{35/Palmeiras,
				26/São Paulo,
				30/Corinthians,
				9/Santos}
		\end{tikzpicture}
	\end{center}
	
	\begin{center}
		\begin{tikzpicture}[scale=0.5]
			\pie[explode = {0,0,0.4,0}]{35/Palmeiras,
				26/São Paulo,
				30/Corinthians,
				9/Santos}
		\end{tikzpicture}
	\end{center}

\textbf{\uline{Exemplo 15: Gráfico de setores com números ocultos}}
    \begin{center}
		\begin{tikzpicture}[scale=0.5]
			\pie[hide number]{35/Palmeiras,
				26/São Paulo,
				30/Corinthians,
				9/Santos}
		\end{tikzpicture}
	\end{center}

\textbf{\uline{Exemplo 15: Gráfico de setores com texto dentro de cada setor}}
    \begin{center}
		\begin{tikzpicture}[scale=0.5]
			\pie[text = inside]{35/Palmeiras,
				26/São Paulo,
				30/Corinthians,
				9/Santos}
		\end{tikzpicture}
	\end{center}

\textbf{\uline{Exemplo 15: Gráfico de setores com adição de legenda}}
    \begin{center}
		\begin{tikzpicture}[scale=0.5]
			\pie[text = legend]{35/Palmeiras,
				26/São Paulo,
				30/Corinthians,
				9/Santos}
		\end{tikzpicture}
	\end{center}

\textbf{\uline{Exemplo 15: Gráfico de setores com adição de pins}}
    \begin{center}
		\begin{tikzpicture}[scale=0.5]
			\pie[text = pin]{35/Palmeiras,
				26/São Paulo,
				30/Corinthians,
				9/Santos}
		\end{tikzpicture}
	\end{center}

\textbf{\uline{Exemplo 15: Gráfico de setores com alteração de fonte}}
    \begin{center}
		\begin{tikzpicture}[scale=0.5]
			\pie[scale font]{35/Palmeiras,
				26/São Paulo,
				30/Corinthians,
				9/Santos}
		\end{tikzpicture}
	\end{center}
	
\textbf{\uline{Exemplo 15: Gráfico de setores com plotagem sem transformar em porcentagem}}
    \begin{center}
		\begin{tikzpicture}[scale=0.5]
			\pie[sum = auto]{35/Palmeiras,
				26/São Paulo,
				30/Corinthians,
				9/Santos}
		\end{tikzpicture}
	\end{center}
	
\textbf{\uline{Exemplo 15: Gráfico de setores com plotagem parcial de setores}}
\begin{center}
    \begin{tikzpicture}[scale=0.5]
    \pie[sum = 80]
        {35/Palm.,
        26/SP}
    \end{tikzpicture}
\end{center}


\textbf{\uline{Exemplo 15: Gráfico de setores com formato polar}}
\begin{center}
    \begin{tikzpicture}[scale=0.5]
    \pie[sum = auto,polar]
        {8/Palm.,
            6/SP,
            7/Cor.,
        2/San.}
\end{tikzpicture} 
\end{center}

\textbf{\uline{Exemplo 15: Gráfico de setores com formato quadrado}}
\begin{center}
    \begin{tikzpicture}[scale=0.5]
    \pie[sum = auto,square]
        {8/Palm.,
            6/SP,
            7/Cor.,
        2/San.}
\end{tikzpicture} 
\end{center}

\textbf{\uline{Exemplo 15: Gráfico de setores com formato nuvem}}
\begin{center}
    \begin{tikzpicture}[scale=0.5]
    \pie[sum = auto,cloud]
        {8/Palm.,
            6/SP,
            7/Cor.,
        2/San.}
\end{tikzpicture} 
\end{center}
\newpage
\textbf{\uline{Exemplo 15: Múltiplos gráficos de setores com uma só legenda}}
\begin{center}
    \begin{tikzpicture}[scale=0.7]
        \pie[pos={0,0},radius=1.5,sum=auto,text=]
            {22/Maça,5/Banana}
        \pie[pos={4,0},radius=1.5,sum=auto,text=]
            {22/Maça,5/Banana,8/Abacaxi}
        \pie[pos={8,0},radius=1.5,sum=auto,text=legend]
            {22/Maça,5/Banana,10/Laranja,8/Pêssego}
    \end{tikzpicture}
\end{center}




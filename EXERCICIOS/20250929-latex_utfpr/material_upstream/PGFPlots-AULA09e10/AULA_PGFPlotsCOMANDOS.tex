% -----------------------------------------------------------------------------
\documentclass[11pt,a4paper]{article}	% Tipo de documento -> artigo + fonte tamanho 11 + escrito em folha de papel A4
\usepackage[brazil]{babel}										% Idioma a ser usado
\usepackage[utf8]{inputenc}										% Escrita de caracteres acentuados e cedilhas - 1
\usepackage[T1]{fontenc}	% Escrita de caracteres acentuados + outros detalhes técnicos fundamentais
% ---------------------------------------
\usepackage{geometry}		% Package para definição das margens do documento
\geometry{a4paper,left=1cm,right=1cm,bottom=3cm,top=2cm,headsep=1cm,footskip=2.5cm}	% Configuração das margens do documento
% ---------------------------------------
\usepackage{graphicx,float,color,xcolor,caption}					% Inserir imagens
\usepackage{ulem,verbatim}									% Escrita literal + sublinhado
\usepackage{amsmath,amsfonts,amssymb,amsthm,amstext}	
\DeclareMathOperator{\sen}{sen}
% ---------------------------------------
\usepackage{tikz,pgfplots,pst-all}					% Desenho vetorial
\usepackage{animate}
\usepackage{pgf-pie}
\pgfplotsset{compat=1.17}
\usepgfplotslibrary{fillbetween}
\usepgfplotslibrary{polar}
% -----------------------------------------------------------------------------
\renewcommand{\baselinestretch}{1}								% Espaçamento entre linhas
% ---------------------------------------
\usepackage{indentfirst,setspace}			% Package para recuo do primeiro parágrafo. Por padrão, o LaTeX só recua a partir do segundo parágrafo
\setlength{\parindent}{1.375cm}				% Recuo do parágrafo -> coloquei 1.375cm
% ---------------------------------------
\usetikzlibrary{external}                   % Para externalização dos gráficos e agilizar a compilação
\tikzexternalize[prefix=tikz/]              %
% ---------------------------------------
% Definição de título, autor e data
\title{Primeiros passos com o \LaTeX,\\ rumo ao WORDEXIT (Word + Exit)\\ Brincando com o PGFplots}				% Título
\author{Angelo Aliano Filho}																% Autor
\date{\today}															% Data
% ---------------------------------------
% Início do documento
\begin{document}

\maketitle

    % \textbf{\uline{Exemplo 01:}}
	\begin{center}
		\begin{tikzpicture}[scale=0.8]
			\begin{axis}[
				xmin = 0, xmax = 30,
				ymin = -1.5, ymax = 2.0,
				%axis x line*=middle,
				xtick distance = 2.5,
				ytick distance = 0.5,
				grid = both,
				xlabel=$x$,
				ylabel=$y$,
				minor tick num = 1,
				major grid style = {lightgray},
				minor grid style = {lightgray!25},
				%legend style={at={(1.03,1)},anchor=north west,legend columns = 1},
				legend style={at={(0.5,1.03)},anchor=south,legend columns = 2},
				width = \textwidth,
				height = 0.45\textwidth]
				\addplot[
				domain = 0:30,
				samples = 200,
				smooth,
				thick,
				blue,
				] {exp(-x/10)*( cos(deg(x)) + sin(deg(x))/10 )};
				\addplot[
				domain = 0:30,
				samples = 200,
				smooth,
				thick,
				red,
				dashed,
				] {cos(0.5*deg(x))};
				\legend{$f(x)=e^{-x/10}(\cos x + \sin x)$,$g(x)=\cos(x/2)$}
			\end{axis}
		\end{tikzpicture}
	\end{center}1
    % \textbf{\uline{Exemplo 02:}}

	\begin{center}
		\begin{tikzpicture}[font=\scriptsize]
			\begin{axis}[thick,smooth,axis lines=middle,width=0.9\textwidth,
				height=0.5\textwidth,
				xmin=-5,xmax=5.2,ymin=-5,ymax=5.2,
				every axis x label/.style={
					at={(ticklabel* cs:1.02)}},
				every axis y label/.style={
					at={(ticklabel* cs:1.02)}},
				xlabel=$x$,ylabel=$y$,ytick distance=1,samples=200,smooth]
				\addplot[blue,domain=-5:-1.01] {(x-1.5)/(x^2-1) + 1};
				\addplot[blue,domain=-0.99:0.99] {(x-1.5)/(x^2-1) + 1};
				\addplot[blue,domain=1.01:5] {(x-1.5)/(x^2-1) + 1};
				\addplot[densely dashed,domain=-5:5] ({1},{x});
				\addplot[densely dashed,domain=-5:5] ({-1},{x});
				\addplot[densely dashed,domain=-5:5] {1};
				\addplot[mark=*,mark size=2pt,fill=white] coordinates {(2,1.17)};
				\node[label={90:{$(2,f(2))$}}] at (axis cs:2,1.17) {};
			\end{axis}
		\end{tikzpicture}
	\end{center}
    % \textbf{\uline{Exemplo 03:}}
	\begin{center}
		\begin{tikzpicture}[font=\scriptsize]
			\begin{axis}[thick,smooth,axis lines=middle,width=0.9\textwidth,
				height=0.5\textwidth,
				xmin=-6.5,xmax=6.55,ymin=-1.5,ymax=1.5,
				every axis x label/.style={
					at={(ticklabel* cs:1.03)}},
				every axis y label/.style={
					at={(ticklabel* cs:1.05)}},
				xtick={-6.28318, -4.7123889, ..., 6.28318},
				xticklabels={$-2\pi$, $-\frac{3\pi}{2}$, $-\pi$   , $-\frac{\pi}{2}$,,  $\frac{\pi}{2}$,  $\pi$   , $\frac{3\pi}{2}$, $2\pi$},
				xlabel=$x$,ylabel=$y$,ytick distance=0.5,samples=400]
				\addplot[blue,domain=-2*pi:2*pi] {cos(deg(x))} node[above,xshift=-0.5cm]{$\cos x$};
				\addplot[red,domain=-2*pi:2*pi] {sin(deg(x))} node[above,xshift=-0.5cm]{$\sen x$};
			\end{axis}
		\end{tikzpicture}
	\end{center}
    % \uline{\textbf{Exemplo 04:}}
\begin{center}
   \begin{figure}[!htb]
   \centering 
	\begin{tikzpicture}[scale = 1.3]
		\begin{axis}[axis lines = center, xtick = {-5,-3,...,3,5}, ytick = {-15,-10,-5,5},  xmin = -6, xmax = 6, ymin = -16, ymax = 8, xlabel = $ x $, ylabel = $ y $]
			\addplot[name path = A, domain = -4:4, samples = 200, color = black] {x^2 - 14} node [pos=0, above] {$ y = x^2-14 $};
			\addplot[name path = B, domain = -4:4, samples = 200, color = black] {4 - x^2} node [pos=1, below] {$ y = 4-x^2 $};
			\addplot[yellow!40, opacity = 0.5, domain = -3:3] fill between[of=A and B, soft clip = {domain = -3:3}];
			\addplot[color=black,mark=none, densely dotted]  coordinates {(-3,0) (-3,-5) };
			\addplot[color=black,mark=none, densely dotted]  coordinates {(3,0) (3,-5) };
			\addplot[color=black,mark=none, densely dotted]  coordinates {(-3,-5) (3,-5) };
			\draw [densely dotted] (-3,0) -- (-3,-5);
			\node[above,black] at (axis cs: 1,-7) {\large $ S $};
		\end{axis}
	\end{tikzpicture}
	\caption{Área $ S $ compreendida entre as funções $ y = 4-x^2 $ e $ y = x^2 - 14 $.}
    \end{figure}
\end{center}
    % \textbf{\uline{Exemplo 05:}}
	\begin{center}
		\def\R{1}
		\begin{tikzpicture}[scale=0.8]
			\begin{axis}[
				xmin = -1.5, xmax = 1.5,
				ymin = -1.5, ymax = 1.5,
				xtick distance = 1,
				ytick distance = 1,
				ylabel near ticks,
				xlabel=$x$,
				ylabel=$y$,
				samples=401,
				width = 0.6\textwidth,
				height = 0.6\textwidth]
				\addplot[densely dotted,thick]  
				({\R*sin(deg(x))},{\R*cos(deg(x))});
				
				\foreach \r/\w in {0.1/10,0.25/30,0.2/50,0.125/90}{
					\edef\temp{\noexpand%
						\addplot[thick,
						color=red!\w!blue,
						] (
						{(\R-\r)*cos(deg(x)) + \r*cos(deg((\R-\r)*x/\r))},
						{(\R-\r)*sin(deg(x)) - \r*sin(deg((\R-\r)*x/\r))}
						);
					}\temp
				}
			\end{axis}
		\end{tikzpicture}
	\end{center}
    % \textbf{\uline{Exemplo 06:}}
	\begin{center}
		\begin{tikzpicture}[scale=0.8]
			\begin{axis}[xlabel={$x$},ylabel={$y$},zlabel={$z$},
				view={45}{30},samples=30,samples y=30]
				\addplot3[surf,
				%	shader = interp,
				samples = 60,
				samples y = 60,
				domain = -2:2,
				domain y = -2:2,
				colormap/bluered,
				]
				({x}, {y}, {x*y*exp(-x^2-y^2)});
			\end{axis}
		\end{tikzpicture}
	\end{center}
     \textbf{\uline{Exemplo 07:}}
	\begin{center}
		\begin{tikzpicture}[scale=0.8]
			\begin{polaraxis}[xlabel=$r$,ylabel=$\theta$,ytick distance=0.5,xtick distance=15]
				\addplot+[color=cyan!50!white,mark=none,domain=-360:360,samples=600,line width=2] 
				(x,{cos(2*x)}); 
				\addplot+[color=magenta!50!white,mark=none,domain=-360:360,samples=600,line width=2] 
				(x,{1/2});
				\addplot+[color=cyan!50!white,mark=none,domain=-30:30,samples=600,line width=2,name path=A] 
				(x,{cos(2*x)}); 
				\addplot+[color=magenta!50!white,mark=none,domain=-30:30,samples=600,line width=2,name path=B] 
				(x,{1/2});
				\tikzfillbetween[of=A and B]{opacity = 0.5,color=cyan!80!white}; 
			\end{polaraxis}
		\end{tikzpicture}
	\end{center}
	% \textbf{\uline{Exemplo 08:}}
	\begin{center}
        \begin{tikzpicture}
        	\begin{axis}[
        		xlabel=$x$,
        		ylabel=$y$,
        		width=10cm,
        		height=8cm,
        		axis lines = middle,
        		xmin=-2,
        		xmax=2,
        		ymin=-1,
        		%ymax=6000,
        		y tick label style={/pgf/number format/.cd,
        			set thousands separator={.}
        		},
        		legend style = {
        			at={(0,1)},
        			anchor=north west,
        			font=\scriptsize,
        			legend columns=1,
        			legend cell align=left,
        			draw=none
        		},
        		]
        		\draw[fill] (axis cs: 0,1) circle (2pt);
        		\node[anchor=south west] at (axis cs: 0,1) {$P$};
        		\draw[densely dotted] (axis cs: 0.5,0) -- (axis cs: 0.5,4) -- (axis cs: 0,4);
        		\draw[fill] (axis cs: 0.5,4) circle (2pt);
        		%
        		\foreach \a [count=\i] in {1,1.2,...,3}{
        			\edef\grafico{\noexpand
        				\addplot[blue!\fpeval{(\i-1)*100/10}!red,line width=1pt,smooth,samples=100,domain=-2:2] ({x},{\a^x});
        			}
        			\addlegendentryexpanded{$a=\fpeval{round(\a,1)}$}
        			\grafico
        		}
        	\end{axis}    
        \end{tikzpicture}
	\end{center}
	% \textbf{\uline{Exemplo 09:}}
	\begin{center}
		\pgfplotstableread{Dados_covid_Brasil_EUA.dat}{\table}
		\begin{tikzpicture}[scale=0.65]
			\begin{axis}[title={Total acumulado de mortes por COVID-19},
				ylabel={mortes acumuladas},
				xlabel={dias após 01/01},
				xmin=1,xmax=14,
				ylabel near ticks,
				enlargelimits=0.05,
				xtick distance=2,ytick distance = 50000,
				ymajorgrids,
				major grid style = {lightgray,densely dashed},
				width = \textwidth,
				height = 0.75\textwidth,
				legend pos = north west
				]
				\addplot[blue,mark=square*] table [x = {dia}, y = {mortesB}] {\table};
				\addplot[red,mark=*] table [x = {dia}, y = {mortesE}] {\table};
				\legend{Mortes no Brasil,Mortes nos EUA}
			\end{axis}
		\end{tikzpicture}
	\end{center}
	% \textbf{\uline{Exemplo 10:}}
\begin{center}
		\pgfplotstableread{Dados_covid_Brasil_EUA.dat}{\dados}
		\begin{tikzpicture}[scale=0.65]
			\begin{axis}[title={Total acumulado de mortes por COVID-19},
				ybar=0.0cm,ymax=650000,
				bar width = 12pt,
				enlargelimits=0.05,
				xtick distance = 1,
				ymajorgrids,
				scaled y ticks = false,
				ylabel near ticks,
				scaled y ticks=base 10:0,
				ylabel={mortes acumuladas},
				width = 1.3\textwidth,
				height = 0.75\textwidth,
				nodes near coords,
				every node near coord/.append style={rotate=90, anchor=west},
				symbolic x  coords={,01/01,08/01,15/01,22/01,29/01,05/02,12/02,19/02,26/02,03/03,10/03,17/03,24/03,31/03,},
				legend pos = north west,
				x tick label style={rotate=90},
				]
				\addplot[blue,fill] table [x = {data}, y = {mortesB}] {\dados};
				\addplot[red,fill] table [x = {data}, y = {mortesE}] {\dados};
				\legend{Mortes no Brasil,Mortes nos EUA}
			\end{axis}
		\end{tikzpicture}
	\end{center}	
	% \textbf{\uline{Exemplo 11:}}
	\begin{center}
		\pgfplotstableread{Dados_covid_Brasil_EUA.dat}{\dados}
		\begin{tikzpicture}[scale=0.65]
			\begin{axis}[xbar=0.0cm, xmax=610000, bar width = 5pt, ytick distance = 1, xlabel near ticks, every node near coord/.append style={font=\small}, scaled x ticks=base 10:0, xlabel={Mortes acumuladas}, symbolic y coords={,01/01,08/01,15/01,22/01,29/01,05/02,12/02,19/02,26/02,03/03,10/03,17/03,24/03,31/03,}, legend pos = south east, x tick label style={rotate=90}, title={Total acumulado de mortes por COVID-19}, width = 1.3\textwidth, height = 0.75\textwidth, nodes near coords,
				%ybar=0.0cm,ymax=650000, %enlargelimits=0.05, %ymajorgrids, %scaled y ticks = false, %ylabel={mortes acumuladas},
			]
				\addplot[blue,fill] table [x = {mortesB}, y = {data}] {\dados};
				\addplot[red,fill] table [x = {mortesE}, y = {data}] {\dados};
				\legend{Mortes no Brasil, Mortes nos EUA}
			\end{axis}
		\end{tikzpicture}
	\end{center}
	% \textbf{\uline{Exemplo 12:}}
	\begin{center}
		\begin{tikzpicture}
			\begin{semilogyaxis}[scale=0.8,
				xlabel=Custo,
				ylabel=Erro]
				\addplot[color=blue,mark=diamond] coordinates {
					(5,    8.31160034e-02)
					(17,   2.54685628e-02)
					(49,   7.40715288e-03)
					(129,  2.10192154e-03)
					(321,  5.87352989e-04)
					(769,  1.62269942e-04)
					(1793, 4.44248889e-05)
					(4097, 1.20714122e-05)
					(9217, 3.26101452e-06)
				};
				\legend{Caso 1}
			\end{semilogyaxis}
		\end{tikzpicture}
	\end{center}
	% \textbf{\uline{Exemplo 13:}}
	\begin{center}
		\begin{tikzpicture}
			\begin{loglogaxis}[scale=0.8,
				xlabel=Custo,
				ylabel=Erro]
				\addplot[color=blue,mark=*] coordinates {
					(5,    8.31160034e-02)
					(17,   2.54685628e-02)
					(49,   7.40715288e-03)
					(129,  2.10192154e-03)
					(321,  5.87352989e-04)
					(769,  1.62269942e-04)
					(1793, 4.44248889e-05)
					(4097, 1.20714122e-05)
					(9217, 3.26101452e-06)
				};
				\legend{Caso 1}
			\end{loglogaxis}
		\end{tikzpicture}
	\end{center}
	% \textbf{\uline{Exemplo 14:}}
	\begin{center}
		\begin{tikzpicture}
			\begin{semilogxaxis}[scale=0.8, xlabel=Custo, nodes near coords, ylabel near ticks, every node near coord/.append style = {rotate = 60, anchor = west, font = \tiny}, ymax=0.12,  xmax = 40000,  ylabel=Erro]
				\addplot[color=blue,mark=*] coordinates {
					(5,    8.31160034e-02)
					(17,   2.54685628e-02)
					(49,   7.40715288e-03)
					(129,  2.10192154e-03)
					(321,  5.87352989e-04)
					(769,  1.62269942e-04)
					(1793, 4.44248889e-05)
					(4097, 1.20714122e-05)
					(9217, 3.26101452e-06)
				};
				\legend{Caso 1}
			\end{semilogxaxis}
		\end{tikzpicture}
	\end{center}
    % \textbf{\uline{Exemplo 15: Gráfico de setores padrão}}	
	\begin{center}
		\begin{tikzpicture}[scale=0.5]
			\pie{35/Palmeiras,
				26/São Paulo,
				30/Corinthians,
				9/Santos}
		\end{tikzpicture}
	\end{center}
	
\textbf{\uline{Exemplo 15: Gráfico de setores com raio alterado}}
    \begin{center}
		\begin{tikzpicture}[scale=0.5]
			\pie[radius = 3.3, rotate = 90]{35/Palmeiras,
				26/São Paulo,
				30/Corinthians,
				9/Santos}
		\end{tikzpicture}
	\end{center}
	
\textbf{\uline{Exemplo 15: Gráfico de setores com mudança de cores}}
    \begin{center}
		\begin{tikzpicture}[scale=0.5]
			\pie[color = {yellow, red, blue, green}]{35/Palmeiras,
				26/São Paulo,
				30/Corinthians,
				9/Santos}
		\end{tikzpicture}
	\end{center}
	
\textbf{\uline{Exemplo 15: Gráfico de setores com setores deslocados}}
    \begin{center}
		\begin{tikzpicture}[scale=0.5]
			\pie[explode = 0.1]{35/Palmeiras,
				26/São Paulo,
				30/Corinthians,
				9/Santos}
		\end{tikzpicture}
	\end{center}
	
	\begin{center}
		\begin{tikzpicture}[scale=0.5]
			\pie[explode = {0,0,0.4,0}]{35/Palmeiras,
				26/São Paulo,
				30/Corinthians,
				9/Santos}
		\end{tikzpicture}
	\end{center}

\textbf{\uline{Exemplo 15: Gráfico de setores com números ocultos}}
    \begin{center}
		\begin{tikzpicture}[scale=0.5]
			\pie[hide number]{35/Palmeiras,
				26/São Paulo,
				30/Corinthians,
				9/Santos}
		\end{tikzpicture}
	\end{center}

\textbf{\uline{Exemplo 15: Gráfico de setores com texto dentro de cada setor}}
    \begin{center}
		\begin{tikzpicture}[scale=0.5]
			\pie[text = inside]{35/Palmeiras,
				26/São Paulo,
				30/Corinthians,
				9/Santos}
		\end{tikzpicture}
	\end{center}

\textbf{\uline{Exemplo 15: Gráfico de setores com adição de legenda}}
    \begin{center}
		\begin{tikzpicture}[scale=0.5]
			\pie[text = legend]{35/Palmeiras,
				26/São Paulo,
				30/Corinthians,
				9/Santos}
		\end{tikzpicture}
	\end{center}

\textbf{\uline{Exemplo 15: Gráfico de setores com adição de pins}}
    \begin{center}
		\begin{tikzpicture}[scale=0.5]
			\pie[text = pin]{35/Palmeiras,
				26/São Paulo,
				30/Corinthians,
				9/Santos}
		\end{tikzpicture}
	\end{center}

\textbf{\uline{Exemplo 15: Gráfico de setores com alteração de fonte}}
    \begin{center}
		\begin{tikzpicture}[scale=0.5]
			\pie[scale font]{35/Palmeiras,
				26/São Paulo,
				30/Corinthians,
				9/Santos}
		\end{tikzpicture}
	\end{center}
	
\textbf{\uline{Exemplo 15: Gráfico de setores com plotagem sem transformar em porcentagem}}
    \begin{center}
		\begin{tikzpicture}[scale=0.5]
			\pie[sum = auto]{35/Palmeiras,
				26/São Paulo,
				30/Corinthians,
				9/Santos}
		\end{tikzpicture}
	\end{center}
	
\textbf{\uline{Exemplo 15: Gráfico de setores com plotagem parcial de setores}}
\begin{center}
    \begin{tikzpicture}[scale=0.5]
    \pie[sum = 80]
        {35/Palm.,
        26/SP}
    \end{tikzpicture}
\end{center}


\textbf{\uline{Exemplo 15: Gráfico de setores com formato polar}}
\begin{center}
    \begin{tikzpicture}[scale=0.5]
    \pie[sum = auto,polar]
        {8/Palm.,
            6/SP,
            7/Cor.,
        2/San.}
\end{tikzpicture} 
\end{center}

\textbf{\uline{Exemplo 15: Gráfico de setores com formato quadrado}}
\begin{center}
    \begin{tikzpicture}[scale=0.5]
    \pie[sum = auto,square]
        {8/Palm.,
            6/SP,
            7/Cor.,
        2/San.}
\end{tikzpicture} 
\end{center}

\textbf{\uline{Exemplo 15: Gráfico de setores com formato nuvem}}
\begin{center}
    \begin{tikzpicture}[scale=0.5]
    \pie[sum = auto,cloud]
        {8/Palm.,
            6/SP,
            7/Cor.,
        2/San.}
\end{tikzpicture} 
\end{center}
\newpage
\textbf{\uline{Exemplo 15: Múltiplos gráficos de setores com uma só legenda}}
\begin{center}
    \begin{tikzpicture}[scale=0.7]
        \pie[pos={0,0},radius=1.5,sum=auto,text=]
            {22/Maça,5/Banana}
        \pie[pos={4,0},radius=1.5,sum=auto,text=]
            {22/Maça,5/Banana,8/Abacaxi}
        \pie[pos={8,0},radius=1.5,sum=auto,text=legend]
            {22/Maça,5/Banana,10/Laranja,8/Pêssego}
    \end{tikzpicture}
\end{center}




	 %\textbf{\uline{Exemplo 16:}}
\begin{figure}[H]
\centering 
\animategraphics
[controls={step,stop,play},buttonsize=5mm,scale=0.85,controls={step,stop,play}]
{10}%frame rate
{hipercicloide}%nome do ficheiro que tem as imagens
{}%primeiro frame
{}%último frame
\end{figure}
	
	% \textbf{\uline{Exemplo 17:}}
	\begin{center}
		\begin{tikzpicture}[scale = 1.0]
			\begin{axis}[axis lines = center, xtick = {-4,0,2}, ytick = {0,8}, xmin = -5, xmax = 3, ymin = -2, ymax = 10, xlabel = $ x $, ylabel = $ y $]
				\addplot[name path = A, domain = -5:3, samples = 200, color = black] {x+6};
				\addplot[name path = B, domain = -5:3, samples = 200, color = black] {-0.5*x};
				\addplot[name path = C, domain = -2:3, samples = 200, color = black] {x^3};
				\addplot[color=black,mark=none, densely dotted]  coordinates {(2,0) (2,8) };
				\addplot[color=black,mark=none, densely dotted]  coordinates {(0,8) (2,8) };
				\addplot[color=black,mark=none, densely dotted]  coordinates {(-4,0) (-4,2) };
				\addplot[yellow!20] fill between[of=A and B, soft clip = {domain = -4:0}]; 
				\addplot[green!20]  fill between[of=A and C, soft clip = {domain = 0:2}];
				\node[above,black] at (axis cs: -1.5,2) {$ S_1 $};
				\node[above,black] at (axis cs: 0.7,3.2) {$ S_2 $};
				\node[above] at (axis cs: -2.8,3.9) {\footnotesize $ y = x + 6 $};
				\node[above] at (axis cs: -3,0.1) {\footnotesize $ y = -0,5x $};
				\node[above] at (axis cs: 2.2,3) {\footnotesize $ y = x^3 $}; 
			\end{axis}
		\end{tikzpicture}
		\captionof{figure}{Área $ S $ compreendida entre as funções $ y = x+6 $, $ y = -0.5x $ e $ y = x^3 $.}
		\label{fig: Prova_questaoUM_a}
	\end{center}
	% \textbf{\uline{Exemplo 18:}}
	\begin{center}
		\begin{tikzpicture}[font=\scriptsize]
			\begin{axis}[thick,smooth,axis lines=middle,width=0.5\textwidth,
				height=0.5\textwidth,
				xmin=0,xmax=4,ymin=0,ymax=2,
				every axis x label/.style={
					at={(ticklabel* cs:1.05)}},
				every axis y label/.style={
					at={(ticklabel* cs:1.05)}},
				xlabel=$x$,ylabel=$y$,ytick distance=0.5]
				\addplot[samples=300,name path=A,black,domain=0:4] {sqrt(x)} node[pos=0.5,above,sloped]{$f(x)=\sqrt{x}$};
				\addplot[samples=300,name path=B,black,domain=0:4] {sqrt(x/2)} node[pos=0.5,below,sloped]{$g(x)=\sqrt{x/2}$};
				\addplot[blue!50,opacity=0.5] fill between[of=A and B];
			\end{axis}
		\end{tikzpicture}
	\end{center}
	% \textbf{\uline{Exemplo 19:}}
	\begin{center}
		\begin{tikzpicture}
			[declare function = {
				f(\x) = \x;
				g(\x) = 0;
				F(\x,\y) = 0.5*(\x^2+\y^2);
			}]
			\begin{axis}
				[ xlabel=$x$,scale=0.8,
				ylabel=$y$,zlabel=$z$,
				% axis equal,
				zmin=0,
				zmax=1.5,xmin=0,ymin=0,xmax=1,ymax=1,
				domain=0:1, y domain=0:1,
				samples=50, samples y=10,
				variable y=t,
				view={-30}{20},no marks
				]
				\addplot3[surf,gray,opacity=0.5,fill opacity=0.5,faceted color=gray]       (x, {f(x)*t+g(x)*(1-t)}, 0);
				\addplot3[surf,blue!20,opacity=0.5,fill opacity=0.5,faceted color=blue!20] (x, {f(x)*t+g(x)*(1-t)}, { F(x,f(x)*t+g(x)*(1-t))});  
				\addplot[name path=A,domain=0:1,black!50,line width=1.] ({x},{f(x)});
				\addplot[name path=B,domain=0:1,black!50,line width=1.] ({x},{g(x)});
				\addplot3[name path=C,no marks,samples y=0,domain=0:1,black!50,line width=1.] ({x},{f(x)},{F(x,f(x))});
				\addplot3[name path=D,no marks,samples y=0,domain=0:1,black!50,line width=1.] ({x},{g(x)},{F(x,g(x))});
				\addplot[blue!50,opacity=0.5] fill between [of=A and C];
				\addplot[blue!50,opacity=0.5] fill between [of=B and D];
			\end{axis}
		\end{tikzpicture}
	\end{center}
	% \textbf{\uline{Exemplo 20:}}
	\begin{center}
		\begin{tikzpicture}  
			\begin{axis} 
				[ybar,enlargelimits=0.25,width=\textwidth, 
				height=0.5\textwidth,ybar=0.5cm,
				legend style={at={(0.5,-0.15)},    
					anchor=north,legend columns=3},     
				ylabel={\#Crescimento anual},
				symbolic x coords={2016, 2017, 2018},  
				xtick=data,  
				nodes near coords,  
				nodes near coords align={above}]  
				\addplot[fill=orange,draw=orange] coordinates {(2016, 75) (2017, 78) (2018, 80)};
				\addplot[fill=yellow] coordinates {(2016, 70) (2017, 63) (2018, 68)};  
				\addplot[fill=red] coordinates {(2016, 61) (2017, 55) (2018, 59)};  
				\legend{Trigo, Chá, Arroz}  
			\end{axis}  
		\end{tikzpicture}  
	\end{center}
    % \textbf{\uline{Exemplo 21:}}
\begin{center}
	\begin{tikzpicture}[scale = 1.0]
		\centering
		\begin{axis}[axis lines = center, xtick = {-2,0,2}, ytick = {0,1}, xmin = -3, xmax = 3, ymin = -1, ymax = 5.5, xlabel = $ x $, ylabel = $ y $]
			\addplot[name path = A, domain = -3:3, samples = 200, color = black] {2^x};
			\addplot[name path = B, domain = -3:3, samples = 200, color = black] {2^(-x)};
			\addplot[name path = C, domain = -3:3, samples = 200, color = black] {4};
			\addplot[name path = D, domain = -2:2, samples = 200, color = black] {0};
			\addplot[color=black,mark=none, densely dotted]  coordinates {(-2,0) (-2,4) };
			\addplot[color=black,mark=none, densely dotted]  coordinates {(2,0) (2,4) };
			\addplot[yellow!20, domain = -2:0] fill between[of=C and B, soft clip = {domain = -2:0}];
			\addplot[yellow!20, domain =  0:2] fill between[of=C and A, soft clip = {domain = 0:2}];
			\addplot[green!20,  domain = -2:0] fill between[of=B and D, soft clip = {domain = -2:0}];
			\addplot[blue!20,   domain =  0:2] fill between[of=A and D, soft clip = {domain = 0:2}];
			\node[above,black] at (axis cs: 0,2.5) {\large $ S $};
			\node[above] at (axis cs: 1.8,4.8) {$ y = 2^{x} $};
			\node[above] at (axis cs: 1,4.1) {$ y = 4 $};
			\node[above] at (axis cs: -1.6,4.8) {$ y = 2^{-x} $};
			\node[above,black] at (axis cs: -1.6,1.5) {$ S_1 $};
			\node[above,black] at (axis cs: 1.6,1.5) { $ S_2 $};
		\end{axis}
	\end{tikzpicture}
	\captionof{figure}{Área $ S $ compreendida entre as funções $ y = 2^x $, $ y = 2^{-x} $ e $ y = 4 $.}
	\label{fig: Questionario_questaoUM_b}	
\end{center}
\end{document}
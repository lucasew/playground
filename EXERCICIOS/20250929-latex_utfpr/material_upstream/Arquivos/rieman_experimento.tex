\documentclass[tikz]{standalone}

\usepackage{amsmath}
\usepackage{amssymb}
\usepackage{tikz}
\usepackage{pgfplots}
\usepackage{color,xcolor}
\usepackage{xfp}
\usetikzlibrary{positioning,math,calc}
\pgfplotsset{compat=1.18}

% ========== PARÂMETROS CONFIGURÁVEIS ==========
\def\maxIters{50}

\def\funcaoExpr#1{-#1*#1 + 4}
\def\funcaoNome{$-x^2 + 4$}
\def\limiteA{-2}
\def\limiteB{2}
\def\integralExata{10.666667}

% Janela de visualização
\def\xmin{-3}
\def\xmax{3}
\def\ymin{-1}
\def\ymax{5}
\def\escala{1.5}

% Espaçamento das marcações nos eixos
\def\deltaX{1}
\def\deltaY{1}

% Posição dos textos informativos (porcentagem de offset das bordas)
\def\textoOffsetX{0.95}  % posição X relativa (0.95 = 95% da largura)
\def\textoOffsetY{0.95}  % posição Y relativa (0.95 = 95% da altura)
% ==============================================

\begin{document}
	
	\pagecolor{black}
	\color{white}
	
	\foreach \n in {2,3,...,\maxIters}{
		
		\tikzmath{
			int \n;
			int \i;
			real \soma;
			\a = \limiteA;
			\b = \limiteB;
			\h = (\b-\a)/\n;
			function F(\x) {
				return \funcaoExpr{\x};
			};
			\soma = 0;
			for \i in {0,1,...,\fpeval{\n-1}} {
				\soma = \soma + \h*F(\a + (\i + 1)*\h); 			
			};
			\exato = \integralExata;
			\erro = 100*abs(\exato - \soma)/\exato;
		}
		
		\begin{tikzpicture}[
			declare function = {f(\x) = \funcaoExpr{\x};},
			scale=\escala, 
			font=\small
			]
			% Define o bounding box (sem clip)
			\path[use as bounding box] (\xmin,\ymin) rectangle (\xmax,\ymax);
			
			\draw[-latex] (\xmin,0) -- (\xmax,0) node[right] {$x$};
			\draw[-latex] (0,\ymin) -- (0,\ymax) node[above] {$y$};
			
			% Marcações no eixo X (excluindo extremos)
			\pgfmathsetmacro{\startX}{ceil(\xmin/\deltaX)*\deltaX + \deltaX}
			\pgfmathsetmacro{\endX}{floor(\xmax/\deltaX)*\deltaX - \deltaX}
			\foreach \x in {\startX,\fpeval{\startX+\deltaX},...,\endX} {
				\pgfmathtruncatemacro{\notZero}{abs(\x) > 0.001 ? 1 : 0}
				\ifnum\notZero=1
				\draw (\x,2pt) -- (\x,-2pt) node[below] {$\pgfmathprintnumber{\x}$};
				\fi
			}
			
			% Marcações no eixo Y (excluindo extremos)
			\pgfmathsetmacro{\startY}{ceil(\ymin/\deltaY)*\deltaY + \deltaY}
			\pgfmathsetmacro{\endY}{floor(\ymax/\deltaY)*\deltaY - \deltaY}
			\foreach \y in {\startY,\fpeval{\startY+\deltaY},...,\endY} {
				\pgfmathtruncatemacro{\notZero}{abs(\y) > 0.001 ? 1 : 0}
				\ifnum\notZero=1
				\draw (2pt,\y) -- (-2pt,\y) node[left] {$\pgfmathprintnumber{\y}$};
				\fi
			}
			
			\draw[cyan,thick,domain=\a:\b,smooth,samples=100] plot (\x, {f(\x)});
			
			\foreach \i in {0,1,...,\fpeval{\n-1}}{
				\draw[magenta,opacity=0.3,fill=magenta] ({\a + \i*\h},{0}) rectangle ({\a + (\i+1)*\h},{f(\a + (\i+1)*\h)});
			}
			
			% Textos com posição parametrizada
			\pgfmathsetmacro{\textX}{\xmin + \textoOffsetX*(\xmax-\xmin)}
			\pgfmathsetmacro{\textY}{\ymin + \textoOffsetY*(\ymax-\ymin)}
			
			\node[anchor=north east, inner sep=2pt] at (\textX,\textY) {
				\begin{tabular}{l}
					\funcaoNome \\
					$A_{\n} = \fpeval{round(\soma,5)}$ \\
					$E_{\n} = \fpeval{round(\erro,5)}\%$
				\end{tabular}
			};
		\end{tikzpicture}
	}
	
\end{document}
\documentclass[12pt,a4paper]{article}
\usepackage[T1]{fontenc}
\usepackage[utf8]{inputenc}
\usepackage[brazil]{babel}
\usepackage{lipsum}

\usepackage{amsmath,amssymb,amsthm,mathtools,stackrel}
\usepackage{polynom}
\usepackage{hyperref}
\hypersetup{
	colorlinks,
	linkcolor=blue,
	urlcolor=cyan,
}

\usepackage{xlop,pst-all} % operações básicas
\usepackage{cancel} % cancelamentos
\usepackage{xfp} % cálculos aritméticos dentro do latex

\author{Lucas Eduardo}
\date{\today}
\title{Trabalho}

\begin{document}
	\maketitle
	
	Fórmulas: $2 + 2 = 5$

	Fórmulas destacadas: $$2 + 2 = 5$$
	
	Fração: $\frac{69}{420}$
	
	Raiz: $\sqrt[420]{69}$
	
	Integral: $\int{ino}$  señores
	
	Sub e super escrito: $x_2^2$
	
	Delimitador: $\left\{ \mbox{\huge BOM DIA} \right\}$


	Imagem no modo matemática $
	\left[{\text{\includegraphics[scale=0.05]{logoUTFPR.png}}}\right]
	$
	
	Numeração da equação \ref{eq_uino}: 
	\begin{equation}
		2 + 2 = 5
		\label{eq_uino}
	\end{equation}
	
\end{document}


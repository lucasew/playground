\documentclass[12pt,a4paper]{article}
\usepackage[T1]{fontenc}
\usepackage[utf8]{inputenc}
\usepackage[brazil]{babel}
\usepackage{lipsum}

\usepackage{graphicx}
\usepackage{float}
\usepackage{subfig}
\usepackage{caption,subcaption}
\usepackage{wrapfig}
\usepackage{lscape} % rotaciona objeto
\usepackage{pdflscape} % rotaciona página
\usepackage{multicol}


\author{Shaolin Matador de Porco}
\title{Tabelas, Figuras e Referências}
\date{\today}

\begin{document}
	\maketitle
\begin{multicols}{2}
	
	
	\section{A}
	
	\lipsum[1]
	\section{B}
	
	\lipsum[1]
\end{multicols}
	
	\begin{figure}[!htb]
		\centering
		\includegraphics[width=0.5\linewidth,height=3cm]{./logoUTFPR.png}
		\includegraphics[scale=0.2]{./logoUTFPR.png}
		\caption{Logo da UTFPR}
		\label{utfpr}
	\end{figure}
\begin{multicols}{2}
	
	\newpage
	\lipsum[1]
\end{multicols}	

	\begin{figure}[!t]
		\begin{minipage}[c][4cm][c]{0.49\textwidth}
			\centering
			\includegraphics[scale=0.25]{./logoUTFPR.png}
		\end{minipage}
		\begin{minipage}[c][4cm][c]{0.49\textwidth}
			\centering
			\includegraphics[scale=0.25]{./logoUTFPR.png}
		\end{minipage}
		\caption{Figuras side by side}
	\end{figure}
\begin{multicols}{2}
	\lipsum[1]
\end{multicols}

	\begin{figure}[!htb]
		\centering
		\subfloat[Logo UTFPR]{\label{fig:utfpr_a}\includegraphics[scale=0.10]{logoUTFPR.png}}\hspace{1cm}\subfloat[Logo UTFPR]{\label{fig:utfpr_b}\includegraphics[scale=0.10]{logoUTFPR.png}}\hspace{1cm}\subfloat[Logo UTFPR]{\label{fig:utfpr_c}\includegraphics[scale=0.10]{logoUTFPR.png}}
		\caption{Logo UTFPR}
		
	\end{figure}
	
	As figuras \ref{fig:utfpr_a}, \ref{fig:utfpr_b} e \ref{fig:utfpr_c} são logos da UTFPR

\end{document}
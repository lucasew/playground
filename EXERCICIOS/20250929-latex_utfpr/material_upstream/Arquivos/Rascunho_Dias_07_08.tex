\documentclass[9pt]{beamer}
%-------------------Packages necessários-------------------
\usepackage[T1]{fontenc}
\usepackage[utf8]{inputenc} %Acentuação pelo teclado
\usepackage[brazil]{babel} 
\usepackage{graphicx}
\usepackage{lipsum}
\usepackage{tikz}
\usepackage{color,xcolor}
\usepackage{amsthm}
\usepackage{pifont}
\newtheorem{teorema}{Teorema}
\usepackage{pdfpages}
%------------------------------------------------------------
%Configurações
%Inserindo o índice no início de cada seção de sua apresentação
\AtBeginSection[]{
	\begin{frame}
		\frametitle{Sumário}
		\tableofcontents[currentsection]
	\end{frame}
}
%-------------------------------------------------------------------------------
%Justificando o conteúdo em TODOS os slides
\usepackage{ragged2e}
\apptocmd{\frame}{}{\justifying}{}
%-------------------------------------------------------------------------------
%Traduzindo para português o ambiente "example"
\uselanguage{portuguese}
\languagepath{portuguese}
\deftranslation[to=portuguese]{Example}{Exemplo}
%-------------------------------------------------------------------------------
%Incluindo um plano de fundo para todos seus slides (uma figura, por exemplo)
%Recorremos ao Tikz para mudar a opacidade da figura
%\tikz \node[opacity=0.1] { insira a figura };

%\usebackgroundtemplate{
	%	\tikz\node[opacity=0.02]{
		%		\includegraphics[width=\paperwidth, height=\paperheight] {logoUTFPR}
		%	};
	%}

% %Outra opção é você mesmo fazer o desenho em tikz!
%  \usebackgroundtemplate{
	%      \tikz{
		%         \shade[top color=yellow!15, bottom color=white] (0,0) rectangle (12.96,10);
		%         \foreach \j in {0,0.3,...,10}{
			%             \draw[gray,opacity=0.2] (0,\j) -- (13,\j);
			%         }
		%         \foreach \i in {0,0.3,...,13}{
			%             \draw[gray,opacity=0.2] (\i,0) -- (\i,10);
			%         }
		%     }
	% }

%  \usebackgroundtemplate{
	%      \tikz{
		%         %\shade[top color=yellow!15, bottom color=white] (0,0) rectangle (12.96,10);
		%         \foreach \j in {0,0.8,...,10}{
			%             \draw[gray,opacity=0.2] (0,\j) -- (13,\j);
			%         }
		%     }
	% }

%Caso queira que a mudança ocorra localmente, colar esta instrução atrás do slide
%-------------------------------------------------------------------------------
%Inserindo mais de um logo e mudando de posição

% \logo{
	%     \tikz[overlay, remember picture] \node[yshift=0.8cm,xshift=-0.5cm] at (current page.30) {\includegraphics[scale=0.03]{logoUTFPR}};
	%     \tikz[overlay, remember picture] \node[yshift=-0.8cm,xshift=0.5cm] at (current page.210) {\includegraphics[scale=0.03]{logoUTFPR}};
	% }
%-------------------------------------------------------------------------------
% %Colocando o título do slide no centro
% \setbeamertemplate{frametitle}[default][center]
% %-----------------------------------------------------------------------------
% %Inserindo a numeração em cada slide
%\setbeamertemplate{footline}[frame number]

% %Removendo os símbolos de navegação
%\setbeamertemplate{navigation symbols}{}
%-----------------------------------------------------------------------------
%Temas e Layouts

%Boadilla,Bergen,Madrid,Antibes,Hannover,CambridgeUS,AnnArbor,Warsaw

%--------------------------------------------------------------------
%Mudando as cores das paletas

\colorlet{cor1}{red!80!black}
\colorlet{cor2}{yellow!80!black}

%Palete 1
%\setbeamercolor{palette primary}{bg=cor1,fg=white}
%Palete 2
%\setbeamercolor{palette secondary}{bg=cor1,fg=white}
%Palete 3
%\setbeamercolor{palette tertiary}{bg=cor2,fg=white}
%Palete 4
%\setbeamercolor{palette quaternary}{bg=cor2,fg=white}
%--------------------------------------------------------------------
%Cor dos itens e enumerates:
%\setbeamercolor{structure}{fg=cor2}
%--------------------------------------------------------------------
%Cor da seções:
%--------------------------------------------------------------------
%\setbeamercolor{section in toc}{fg=cor2}
%--------------------------------------------------------------------
%Cores do fundo e da letra:
%\setbeamercolor{normal text}{bg=green!3,fg=black}
%--------------------------------------------------------------------
%Cor do título do slide, do fundo da barra que o contém e deixá-lo em negrito:
%--------------------------------------------------------------------
%\setbeamercolor{frametitle}{fg=green,bg=blue}
%\setbeamerfont{title}{series=\bfseries}
% %--------------------------------------------------------------------
% %Cor da barra superior de block da letra:
% \setbeamercolor{block title}{bg=cor1, fg=white}
%--------------------------------------------------------------------
%Cor da caixa do block e da letra:
%\setbeamercolor{block body}{bg=cor2!50, fg=black}
%-----------------------------------------------------------------------------
%Escolhendo uma cor e usando-a em toda a estrutura da apresentação
%\definecolor{MinhaCor}{rgb}{0.0, 0.26, 0.15}
%\setbeamercolor{structure}{fg=MinhaCor} 
%---------------->>>>>> IMPORTANTE
%-------------------------------------------------------------------------------% %Estilos de letra
% %serif,helvet,avant,times,bookman,chancery,charter,euler,mathtime,mathptm,mathptmx,newcent,palatino,pifont,utopia
% %-----------------------------------------------------------------------------
%Tema pronto: Tema, Cores e fontes pré-definidas

%Cores:
% %default,albatross,beaver,beetle,crane,dolphin,dove,fly,lily,orchid,rose,seagull,seahorse,whale,wolverine,spruce

%Temas
%infolines,miniframes,split,shadow,sidebar,smoothbar,smoothtree,tree

%Fontes:
% %default,serif,professionalfonts,structurebold,structureitalicserif,structuresmallcapsserif

%\usecolortheme{seagull}

% \usetheme{Berlin}
% \usefonttheme{structurebold}
% \usecolortheme{albatross}
% \setbeamerfont*{frametitle}{size=\huge}
%-----------------------------------------------------------------------------
%Mudando os bullets dos itens
%rectangles,circles,inmargin,rounded
%\useinnertheme{rectangles}

%Customizando você mesmo....

%\usepackage{enumitem}
%\usepackage{enumerate}

% \setitemize{
	% 	label=\textcolor{cor1}{\ding{111}},
	% 	itemindent=-1em,
	% 	leftmargin=0.5cm,
	% 	parsep=0.5cm
	% } 
% \setenumerate{
	% 	label=\textcolor{cor2}{(\arabic*)},
	% 	itemsep=3pt,topsep=-3pt,
	% 	labelsep=5pt
	% }
%
% \newcommand{\MeuEnumerate}[1]{
	% 	\tikz[baseline=(X.base)] \node[draw=cor2,circle,fill=cor2,text=white,inner sep=1pt] (X) {#1};
	% }
%-----------------------------------------------------------------------------
%Mudando as margens do slide
%\setbeamersize{text margin left=0.25cm}  % <- like this
%\setbeamersize{text margin right=0.25cm} % <- like this
%-------------------------------------------------------------------------------
% %Finalmente, mudando o tamanho do slide
% %aspectratio em \documentclass[aspectratio=xx]{beamer} onde xx pode valer:
% %1610, 169, 149, 54, 43 and 32, which stand for the ratios 16:10, 16:9, 14:9, 5:4, 4:3 and 3:2
% %O default é uma lâmina ter  126 mm por 96 mm ---> aspectratio = 43
% %-----------------------------------------------------------------------------

%Dados da apresentação
\title[Aula do Beamer]{Aulas 07 e 08 -- Fazendo Slides e dando adeus ao Power Point}
\author[Aliano, A.F.]{Angelo Aliano Filho}
\date{\today}
\institute[UTFPR]{Universidade Tecnológica Federal do Paraná}
\keywords{Otimização, Pesquisa Operaciona, Matemática Aplicada}
\logo{\includegraphics[scale=0.05]{logoUTFPR.png}}
%----------------------------------------------------------------------------------------------------------------
\usebackgroundtemplate{
\begin{tikzpicture}
	\shade[top color=yellow!30,bottom color=white] (-13, -10) rectangle (13,10);
\end{tikzpicture}
}

\setbeamertemplate{footline}[framenumber]
\setbeamertemplate{navigation symbols}{} % tira navegação

%\usetheme{Warsaw}

\begin{document}
	
	%------------------------------------------------------------------------------------------------
	\begin{frame}
		\titlepage
	\end{frame}
	%------------------------------------------------------------------------------------------------
	\begin{frame}{Agenda do dia}{Tópicos}
		\tableofcontents
	\end{frame}
	%------------------------------------------------------------------------------------------------
	
	%------------------------------------------------------------------------------------------------
	\section{Introdução}
	
	\begin{frame}{\insertsection}
		Conteúdo do slide alinhado verticalmente no centro
		
		\lipsum[1]
	\end{frame}
	%-------------------------------------------------------------------------------
	\section{Revisão de Literatura}
	
	\begin{frame}[t]{\insertsection}
		Conteúdo do slide alinhado verticalmente no topo
		\begin{itemize}
			\item Texto em \alert{alerta}
		\end{itemize}
	\end{frame}
	%-------------------------------------------------------------------------------
	\begin{frame}{\insertsection}
		\begin{block}{Criando blocos}
			Conteúdo a ser destacado num bloco...
		\end{block}
		\begin{example}{Exemplo 1}
			Conteúdo em exemplo
		\end{example}
		\begin{alertblock}{Alerta!}
			Conteúdo de um alerta
		\end{alertblock}
		\vspace{1cm}
		
		Os ambientes block, alertblock, example, frame, entre outros, são exclusivos da classe de documentos beamer.
	\end{frame}
	%------------------------------------------------------------------------------
	\section{Desenvolvimento do trabalho}
	
	\begin{frame}{\insertsection}
		\begin{enumerate}
			\item content
			\item content
			\item content
			\item content
		\end{enumerate}
		
		\begin{itemize}
			\item content
			\item content
			\item content
			\item content
		\end{itemize}
		
		% \begin{enumerate}
			% 	\item[\MeuEnumerate{1}] Item 1
			% 	\item[\MeuEnumerate{2}] Item 2
			% 	\item[\MeuEnumerate{3}] Item 3
			% \end{enumerate}
		
	\end{frame}
	
	
	
	%------------------------------------------------------------------------------------------------
	%Inserindo um PDF externo "Folder_curso_LaTeX" (fora do slide)
	% {
		% 	\setbeamercolor{background canvas}{bg=}
		% 	\includepdf[pages={1}]{Folder_curso_LaTeX.pdf}
		% }
	
	%------------------------------------------------------------------------------------------------
	%Inserindo múltiplas colunas
	
	\begin{frame}{\insertsection}	
		\begin{columns}[c]
			\begin{column}{0.6\textwidth}
				\justifying 
				\lipsum[1]
			\end{column}
			\hfill 
			\begin{column}{0.4\textwidth}
				\centering
				%\includegraphics[scale=0.1]{logoUTFPR}
			\end{column}
		\end{columns}
		\vspace{0.5cm}
		Aqui alinhamos uma figura e um texto, mas poderia ser qualquer coisa	
	\end{frame}
	
	%Outro exemplo
	
	\begin{frame}{\insertsection}
		\begin{columns}[b]
			\begin{column}{0.35\textwidth}
				\begin{table}
					\caption{Tabela ao lado de fig.}
					\begin{tabular}{|cccc|}
						\hline
						1 & 2 & 3 & 4\\
						5 & 6 & 7 & 8\\
						9 & 10 & 11 & 12\\
						\hline
					\end{tabular}
				\end{table}
			\end{column}
			\begin{column}{0.35\textwidth}
				\begin{figure}
					%\includegraphics[scale=0.05]{logoUTFPR}
					\caption{Logo da instituição}
				\end{figure}
			\end{column}
			\begin{column}{0.3\textwidth}
				Algum texto explicativo...Algum texto explicativo...Algum texto explicativo...Algum texto explicativo...
			\end{column}
		\end{columns}
	\end{frame}
	
	%------------------------------------------------------------------------------------------------
	%Controle de aparência
	
	\begin{frame}{\insertsection}
		\begin{itemize}
			\item Item 1 com o comando [<+,->]
			\item Item 2 com o comando [<+,->]
			\item Item 3 
		\end{itemize}
	\end{frame}
	
	\begin{frame}{\insertsection}
		\begin{itemize}
			\item Item 4 (aqui tem um pause)\pause 
			\item Item 5 (aqui tem um pause)\pause 
			\item Item 6 
		\end{itemize}
	\end{frame}
	
	{\setbeamercovered{transparent=10}
		\begin{frame}{\insertsection}
			\begin{itemize}
				\item Item 7 (aqui tem um pause)\pause 
				\item Item 8 (aqui tem um pause)\pause 
				\item Item 9 
			\end{itemize}
			Aqui, usamos uma transparência de 10\%
		\end{frame}
	}
	
	{\setbeamercovered{highly dynamic}
		\begin{frame}{\insertsection}
			\begin{itemize}
				\item Item 10 (aqui tem um pause)\pause 
				\item Item 11 (aqui tem um pause)\pause 
				\item Item 12 (aqui tem um pause)\pause
				\item Item 13 (aqui tem um pause)\pause 
				\item Item 14 (aqui tem um pause)\pause 
				\item Item 15 (aqui tem um pause)\pause 
				\item Item 16 (aqui tem um pause)\pause 
				\item Item 17 (aqui tem um pause)\pause   
			\end{itemize}
			Aqui, usamos uma transparência dinâmica (opção que se colocada no preâmbulo vira global)
		\end{frame}
	}
	%------------------------------------------------------------------------------------------------
	
	%Overlays
	
	\begin{frame}{Não há um primo supremo}{A prova usa \textit{reductio ad absurdum}.}
		\begin{theorem}
			Não há um primo supremo.
		\end{theorem}
		\begin{proof}
			\begin{enumerate}
				\item<1-> Suponha que $p$ seja o maior primo.
				\item<2-> Seja $q$ o produto dos $p$ primeiros primos.
				\item<3-> Então $q + 1$ n]ao é divisível por nenhum deles.
				\item<1-> Mas $q + 1$ é maior do que $1$, então divisível por algum primo não necessariamente entre os $p$ primeiros.\qedhere
			\end{enumerate}
		\end{proof}
		\uncover<4->{A prova usou \textit{reductio ad absurdum}.}
	\end{frame}
	
	\begin{frame}{\insertsection}
		Usando o comando only:
		\begin{itemize}
			\only<1>{\item Item 1}
			\only<2>{\item Item 2}
			\only<3>{\item Item 3}
		\end{itemize}
		\only<2>{Só aparece no segundo clique.
			Repare que o slide é o mesmo!}	
	\end{frame}
	%------------------------------------------------------------------------------------------------
	\begin{frame}{\insertsection}
		Usando o comando uncover:
		\begin{itemize}
			\uncover<1>{\item Item 1}
			\uncover<2>{\item Item 2}
			\uncover<3>{\item Item 3. Perceberam a diferença?}
		\end{itemize}
		\uncover<2>{Só aparece no segundo clique.
			Repare que o slide é o mesmo!}
	\end{frame}
	
	%Mesmo efeito com as linhas em numa tabela!
	
	\begin{frame}{\insertsection}
		\begin{tabular}{ccc}
			\hline 
			\uncover<1,2,3,4>{1 & 2 & 3}\\ 
			\uncover<2,3,4>{4 & 5 & 6}\\
			\uncover<3,4>{7 & 8 & 9}\\
			\uncover<4>{10 & 11 & 12}\\
			\hline 
		\end{tabular}
	\end{frame}
	
	%Mais outro exemplo complicado
	\begin{frame}{\insertsection}
		\begin{columns}[b]
			\uncover<1,2,3>{
				\begin{column}{0.35\textwidth}
					\begin{table}
						\caption{Tabela ao lado de fig.}
						\begin{tabular}{|cccc|}
							\hline
							1 & 2 & 3 & 4\\
							5 & 6 & 7 & 8\\
							9 & 10 & 11 & 12\\
							\hline
						\end{tabular}
					\end{table}
				\end{column}
			}
			\uncover<2,3>{
				\begin{column}{0.35\textwidth}
					\begin{figure}
						%\includegraphics[scale=0.05]{logoUTFPR}
						\caption{Logo da instituição}
					\end{figure}
				\end{column}
			}
			\uncover<3>{
				\begin{column}{0.3\textwidth}
					Algum texto explicativo...Algum texto explicativo...Algum texto explicativo...Algum texto explicativo...
				\end{column}
			}
		\end{columns}
	\end{frame}
	
	%------------------------------------------------------------------------------------------------
	%Quebrando um slide
	
	\begin{frame}[allowframebreaks]{\insertsection}
		\lipsum[1-5]
	\end{frame}
	%------------------------------------------------------------------------------------------------
	
	%Ajustando em um slide
	
	\begin{frame}[shrink]{\insertsection}
		\lipsum[1-5]
	\end{frame}
	
	
	
	%-------------------------------------------------------------------------------
	
	\begin{frame}{\insertsection}
		\begin{enumerate}
			
			\foreach \n in {0.1, 0.2, ..., 1.0}{
				\item{Número: \n}
			}
		\end{enumerate}
	\end{frame}
\end{document}


% \nocite{*}
% \begin{frame}[allowframebreaks]{Referências}
	% 	\bibliographystyle{plainnat}
	% 	\bibliography{exemplo_bib}
	% \end{frame}


%------------------------------------------------------------------------------------------------
\begin{frame}{\insertsection}
	
\end{frame}

